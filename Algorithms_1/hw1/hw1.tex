\documentclass[12pt]{exam}
\usepackage{amsthm}
\usepackage{libertine}
\usepackage[utf8]{inputenc}
\usepackage[margin=1in]{geometry}
\usepackage{amsmath,amssymb}
\usepackage{multicol}
\usepackage[shortlabels]{enumitem}
\usepackage{siunitx}
\usepackage{cancel}
\usepackage{graphicx}
\usepackage{pgfplots}
\usepackage{listings}
\usepackage{tikz}
\usepackage{setspace}

\usepackage[T2A]{fontenc}
\usepackage[utf8]{inputenc}
\usepackage[russian]{babel}


\pgfplotsset{width=10cm,compat=1.9}
\usepgfplotslibrary{external}
\tikzexternalize

\newcommand{\class}{AI Masters Algorithms} 
\newcommand{\examnum}{HomeWork 1} 
\newcommand{\examdate}{\today} 
\newcommand{\timelimit}{}
\doublespacing


\begin{document}
\pagestyle{plain}
\thispagestyle{empty}

\noindent
\begin{tabular*}{\textwidth}{l @{\extracolsep{\fill}} r @{\extracolsep{6pt}} l}
\textbf{\class} & \textbf{Name:} & \textit{Денис Грачев}\\ %Your name here instead, obviously 
\textbf{\examnum} &&\\
\textbf{\examdate} &&\\
\end{tabular*}\\
\rule[2ex]{\textwidth}{2pt}
% ---


\section*{Задача 1}
\subsection*{a}
Возьмем $C = 1, N = 2$. \\
Тогда $\forall n > N: \log(n) > \log(N) = \log(2) = 1$. \\
Следовательно $\exists C = 1, N = 2: \forall n > N, n = n \log (N) \leq C n \log(n)$.\\
Таким образом $n = O(n \log (n))$. 

\subsection*{б}
$
\exists \varepsilon > 0: n \log(n) = \Omega \left( n^{1 + \varepsilon} \right) \Leftrightarrow \\
\exists \varepsilon > 0: n^{1 + \varepsilon} = O \left( n \log(n) \right) \Leftrightarrow \\
\exists \varepsilon, C, N > 0: \forall n > N, n^{1 + \varepsilon} \leq C n \log(n) 
$\\
Пусть существуют такие $\varepsilon, C, N$ для которых это верно.\\
Возьмем $\log$ от обоих частей неравенства $n^{1 + \varepsilon} \leq C n \log(n)$, тогда \\
$
\log \left( n^{1 + \varepsilon} \right) \leq \log \left( C n \log(n) \right) \Leftrightarrow \\
(1 + \varepsilon)\log(n) \leq \log(C) + \log(n) + \log(\log(n)) \Leftrightarrow \\
\varepsilon \log(n) \leq \log(C) + \log(\log(n)) \Leftrightarrow \\
\varepsilon \log(n) - \log(\log(n)) \leq \log(C)
$\\
Возьмем $n = 2^{2^k}$. Тогда $\log(n) = 2^k, \log(\log(n)) = k$, следовательно\\
$ 
\varepsilon 2^k - k \leq \log(C)
$\\
Заметим что при увеличении $k$ на $1$ левая часть увеличивается на 
$ (\varepsilon 2^{k + 1} - (k + 1)) - (\varepsilon 2^k - k) = \varepsilon 2^k - 1$.  
Eсли взять $k = \log(\frac{2}{\varepsilon})$, то после этого при увеличении $k$ на 1, 
левая часть будет увеличиваться хотя бы на $\varepsilon \frac{2}{\varepsilon} - 1 = 1$. 
Таким образом при увеличении k на 1 в какой то момент неравенство перестанет быть верным 
так как справа стоит константа, противоречье.

\section*{Задача 2}
\subsection*{1.а}
Возьмем $f(n) = n \log(n), g(n) = 1$.\\
Тогда т.к. \\
$\forall n > 1, \log(n) < n $ следовательно $ \forall n > 1, n \log(n) \leq n^2 $ следовательно $ f(n) = n \log(n) = O(n^2)$. \\
$g(n) = 1 = \Omega(1)$ т.к. $\forall n: 1 \leq 1$.\\
$g(n) = 1 = O(n)$ т.к. $\forall n: g(n) = 1 \leq n$.\\
Тогда $h(n) = n \log(n)$. Очевидно что $h(n) = n \log(n) = \Theta (n \log(n))$. 

\subsection*{1.б}
Пусть возможно.\\
$g(n) = \Omega(1) \Leftrightarrow 1 = O(n) \Leftrightarrow \exists C_1, N_1: \forall n > N_1, c_1 \leq g(n)$.\\
$h(n) = \Theta(n^3) \Rightarrow \exists C_2, N_2: \forall n > N_2, C_2 n^3 \leq h(n) $\\
Тогда $\forall n > \max(N_1, N_2), C_1 C_2 n^3 \leq g(n)h(n) = f(n) \Rightarrow f(n) = \Omega(n^3)$. \\
Противоречье, так как по условию $f(n) = O(n^2)$, С одной стороны $f(n) \leq c_1n^2$, с другой $f(n) \geq c_2n^3$.

\subsection*{2.a}

Возьмем $f(n) = 0, g(n) = 1$.\\
$\forall n, f(n) = 0 \leq n^2 \Rightarrow f(n) = O(n^2)$.\\
$g(n) = 1 = \Omega(1)$ т.к. $\forall n: 1 \leq 1$.\\
$g(n) = 1 = O(n)$ т.к. $\forall n: g(n) = 1 \leq n$.\\
тогда $h(n) = \frac{f(n)}{g(n)} = 0$.\\
$\forall n, h(n) = 0 \geq 0 \Rightarrow h(n) = \Omega(0)$. \\
Более нижней нижней оценки чем $\Omega(0)$ не существует она достигается при приведенных $f, g$.

\subsection*{2.б}

$f(n) = O(n^2) \Rightarrow \forall n > N_1, f(n) \leq c_1 n^2$ \\
$g(n) = \Omega(1) \Rightarrow \forall n > N_2, g(n) \geq c_2 $ \\
Следовательно $\forall n > \max(N_1, N_2), h(n) = \frac{f(n)}{g(n)}\leq \frac{c_1 n^2}{g(n)} \leq \frac{c_1}{c_2} n^2 \Rightarrow h(n) = O(n^2)$.\\
$O(n^2)$ верхняя оценка $h(n)$ достигается при $f(n) = n^2, g(n) = 1$.

\section*{Задача 3}
\begin{figure}[h]

    {\tt for} (bound = 1; bound < n; bound *= 2 ) \{ \emph{Цикл 1}

        \hspace{4mm} {\tt for} (i = 0; i < bound; i += 1) \{ \emph{Цикл 2}

        \hspace{8mm} {\tt for} (j = 0; j < n; j += 2) \emph{Цикл 3}

        \hspace{12mm} {\tt печать} (``алгоритм'')

        \hspace{8mm} {\tt for} (j = 1; j < n; j *= 2) \emph{Цикл 4}

        \hspace{12mm} {\tt печать} (``алгоритм'')

        \hspace{4mm} \}

    \} 
\end{figure}

\emph{Цикл 3} имеет $\frac{n}{2} = \Theta(n)$ итераций, \emph{Цикл 4} имеет $\log(n) = \Theta(\log(n))$ итераций.\\
$\Theta(n) + \Theta(\log(n)) = \Theta(n)$, соответсвенно то что внутри \emph{Цикл 2} имеет сложность $\Theta(n)$.\\

Рассмотрим \emph{Цикл 1} и \emph{Цикл 2}, количество итераций в \emph{Цикл 2} - bound, bound - удваивается кажлый раз, пусть в было k итераций в \emph{Цикл 1}.\\
Тогда количетсво итераций \emph{Цикл 1} и \emph{Цикл 2} это $1 + 2 + \ldots 2^k = 2^{k + 1} - 1$.
Из условия цикла очевидно что $\frac{n}{2} < 2^k \leq n$,
тогда $ n \leq 2^{k + 1} - 1 \leq 2n$. \\
Следовательно количество итераций \emph{Цикл 1} и \emph{Цикл 2} это $\Theta(n)$. \\
Итого получается что количество слов алгоритм $\Theta(n) \Theta(n) = \Theta(n^2)$.

\section*{Задача}
Обозначим массивы $x_1, x_2, x_3$ и их длины $n_1, n_2, n_3$ соответсвенно.\\

Алгоритм.\\
Заведем $c = i_1 = i_2 = i_3 = 0$.
\begin{figure}[h]
    {\tt Пока} ($i_1 < n_1 | i_2 < n_2 | i_3 < n_3$) \{ 

        \hspace{4mm} {\tt} $c += 1$

        \hspace{4mm} {\tt} $m = min(x_k[i_k]| k \in (1, 2, 3), i_k < n_k )$ 

        \hspace{4mm} {\tt} Если $i_1 < n_1 \:\&\: x_1[i_1] == m$ 

        \hspace{8mm} {\tt} $i_1 += 1$

        \hspace{4mm} {\tt} Если $i_2 < n_2 \:\&\: x_2[i_2] == m$

        \hspace{8mm} {\tt} $i_2 += 1$

        \hspace{4mm} {\tt} Если $i_3 < n_3 \:\&\: x_3[i_3] == m$

        \hspace{8mm} {\tt} $i_3 += 1$

        \hspace{4mm} 

    \} 

    Печать $c$
\end{figure}

Доказательство корректности.\\
$m$ увеличивается на каждом шаге цикла.\\
Так как массивы отсортированы, то элементы массивов которые были равны $m$ на предыдущем шаге увеличатся, 
элементы массивов которые не были равны $m$ на предыдущем шаге больше $m$, таким образом $m$ точно увеличится. 
Таким образом никакое значение $m$ не примет дважды.\\
$m$ примет все возможные значения из данных массивов.\\
Легко заметить, что $i_k$ увеличивается и проходит все значения от $0$ до $n_k$. 
Итерация на которой происходит увеличение $i_k$ соответсвует ситуации когда $m=x_k[i_k]$. 
Значит $m$ проходит все значения массивов.
Таким образом $m$ проходит все значения массивов и никакое значение не проходит дважды. 
Значит $c$ - счетчик уникальных значений $m$ считает количество различных чисел.
Так как на каждой итерации хотя бы один индекс увеличивается, то количество итераций $O(n_1 + n_2 + n_3)$.

\section*{Задача 6}
Обозначим \\
$A(n) = \sum_{i=1}^{n} a_i$\\
$B(n) = \sum_{i=1}^{n} b_i$\\
$F(n) = \sum_{i, j \leq n, i \neq j} a_i b_j$\\
Тогда \\
$F(n + 1) = \sum_{i, j \leq n + 1, i \neq j} a_i b_j = \sum_{i, j \leq n + 1} a_i b_j - \sum_{i, j \leq n + 1, i = j} a_i b_j = 
\sum_{i, j \leq n} a_i b_j - \sum_{i, j \leq, i = j} a_i b_j + \sum_{i \leq n} a_i b_{n+1} + \sum_{j \leq n} a_{n+1} b_j - a_{n+1}b_{n+1} =
F(n) + A(n)b_{n+1} + B(n)a_{n+1} - a_{n+1}b_{n+1}$.

Алгоритм.\\
$F(0) = A(0) = B(0) = 0$\\
$A(n + 1) = A(n) + a_{n + 1}$\\
$B(n + 1) = B(n) + b_{n + 1}$\\
$F(n + 1) = F(n) + A(n)b_{n+1} + B(n)a_{n+1} - a_{n+1}b_{n+1}$\\
Будем хранить $F(n), A(n), B(n)$ и обновлять их при поступлении новых $a_{n+1}, b_{n+1}$.

\section*{Задача 7}
заведем массив $l$ размера $n$ (индексация с 1). \\
$l[i]$ - длина наидленнейщей возрастающей подпоследовательности среди подпоследовательности
$a_i, a_{i + 1}, \ldots a_{n}$, начинающаяся с $a_i$.\\
Тогда $l[n] = 1$.\\
$l[i] = max(1, l[j] + 1 | a_j > a_i)$. \emph{занимает O(n)}.
Таким образом заполняя справа налево мы получим длины наиболее длинных возрастающих подпоследовательностей начинающихся с $i$-ого элемента за $O(n) * O(n) = O(n^2)$.\\
Корректность этого заполнения: 
Так как мы ищем наиболее длинную возрастающую подпоследовательность начиная с $a_i$, 
то нам подходят элементы стоящие после $a_i$ такие что $a_i < a_j$. 
Так как для них задача уже решена, то длина получившейся последовательности будет $l[j] + 1$ и мы перебираем все подходяшие подпоследовательности поэтому найдем длину наидленнейшей.

После этого найдем $i: l[i] = \max(l)$. 
Это по определению индекс первого элемента строго возрастающей подпоследовательности которая имеет максимальную длину.
После этого найдем если $l[i] = 1$, то это последний элемент подпоследовательности,
иначе найдем $j: j > i, l[i] = l[j] + 1, a_j > a_i$. Такой элемент найдется так как мы так строили $l[i]$ и это будет следующий элемент последовательности.
Приравняем $i = j$ и повторим поиск следующего элемента. Поиск элемента занимает $O(n)$, всего элементов $O(n)$, 
значит сложность $O(n^2)$. 
На каждой итерации мы добавляем 1 корректный элемент к искомой последовательности, элемент всегда находится из построения.
Итоговая сложность $O(n^2)$.

 
\end{document}