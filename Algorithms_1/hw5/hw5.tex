\documentclass[12pt]{exam}
\usepackage{amsthm}
\usepackage{libertine}
\usepackage[utf8]{inputenc}
\usepackage[margin=1in]{geometry}
\usepackage{amsmath,amssymb}
\usepackage{multicol}
\usepackage[shortlabels]{enumitem}
\usepackage{siunitx}
\usepackage{cancel}
\usepackage{graphicx}
\usepackage{pgfplots}
\usepackage{listings}
\usepackage{tikz}
\usepackage{setspace}

\usepackage[T2A]{fontenc}
\usepackage[utf8]{inputenc}
\usepackage[russian]{babel}


\pgfplotsset{width=10cm,compat=1.9}
\usepgfplotslibrary{external}
\tikzexternalize

\newcommand{\class}{AI Masters Algorithms} 
\newcommand{\examnum}{Homework 5} 
\newcommand{\examdate}{\today} 
\newcommand{\timelimit}{}
\newcommand{\pluseq}{\mathrel{+}=}
\newcommand{\minuseq}{\mathrel{-}=}
\doublespacing


\begin{document}
\pagestyle{plain}
\thispagestyle{empty}

\noindent
\begin{tabular*}{\textwidth}{l @{\extracolsep{\fill}} r @{\extracolsep{6pt}} l}
\textbf{\class} & \textbf{Name:} & \textit{Денис Грачев}\\
\textbf{\examnum} &&\\
\textbf{\examdate} &&\\
\end{tabular*}\\
\rule[2ex]{\textwidth}{2pt}
% ---

\section*{Task 1}
Отсортируем строки с помощью RadixSort считая что буквы-цифры. 
Тогда из доказанного на лекции сложность сортировки будет $O(nkl)$, 
где $n$ - длина массива, $k$ - длина строк, $l$ - размер алфавита.

\section*{Task 2}
Решим задачу рекурсивно. 
Пусть решаем задачу для отрезка $l, r$. 
Обозначим $m=\frac{l + r}{2}$. 
Проверим тогда значения $a_{m - 1}, a_{m}, a_{m + 1}$.
Рассмотрим возможные варианты
\begin{itemize}
    \item $\nearrow \searrow$ - тогда $a_m$ - максимальный элемент, ответ получен.
    \item $\nearrow \nearrow$ - тогда максимальный элемент находится на отрезке $l, m - 1$, запустим рекурсивно алгоритм для него.
    \item $\searrow \searrow$ - тогда максимальный элемент находится на отрезке $m + 1, r$, запустим рекурсивно алгоритм для него.
\end{itemize}
Таким образом мы каждый раз уполовиниваем размер входа, и на каждом шаге рекурсии тратим $O(1)$ действий.
Итоговая сложность $O(\log (n))$.\\
Быстрее невозможно, так как возможных ответов $n$ и из доказанного на лекции 
чтобы с помощью бинарных вопросов найти в этом случае ответ, необходимо хотя бы $O(\log(n))$ вопросов.

\section*{Task 3}
Решим задачу рекурсивно. 
Разделим на 3 равные кучки и взвесим первые две.
Рассмотрим возможные варианты
\begin{itemize}
    \item Первая кучка легче. Значит фальшивая монета в ней, решим задачу для нее.
    \item Вторая кучка легче. Значит фальшивая монета в ней, решим задачу для нее.
    \item Они равны. Значим фальшивая монетка в 3 кучке, решим задачу для нее.
\end{itemize}
Таким образом после каждого взвешивания подозрительная кучка уменьшается в 3 раза, 
следовательно количество взвешиваний будет $\log_3(n) + c$ из-за округлений.

\section*{Task 4}
Рассмотрим дерево решений. 
У него должно быть хотя бы $n$ листьев, так как возможно $n$ различных ответов. 
Каждая вершина имеет 3 ребенка ($<, >, =$), таким образом количество листьев на слое $h$ это $3^h$.
Следовательно, минимальная необходимая высота это $\log_3(n)$.

\section*{Task 5}
Обозначим массивы $l_1, l_2$, искомую медиану $m$. 
Решим задачу рекурсивно.
Тогда медиана $l_1$ это $l_1[n / 2]$, медиана $l_2$ это $l_2[n / 2]$. 
\begin{itemize}
    \item $l_1[n / 2] < l_2[n / 2]$, тогда отрежем половину у $l_1$ слева, а у $l_2$ спрва и решим задачу рекурсивно.
    Действительно, $l_1[n / 2] \leq m \leq l_2[n / 2]$, иначе с одной стороны от медианы будет больше $n$ значений.
    Так же после отрезания половин, мы убрали $n / 2$ чисел меньших $m$ и $n / 2$ чисел больших $n$, следовательно медиана осталась прежней.
    \item $l_1[n / 2] > l_2[n / 2]$, аналогично наоборот.
    \item $l_1[n / 2] = l_2[n / 2]$ мы нашли медиану, так как тогда одинаковое количество чисел меньше и больше $l_1[n / 2] = l_2[n / 2]$.
\end{itemize}
Таким образом мы найдем медиану. 
Каждый раз длина входа делится на 2, операции стоят $O(1)$ следовательно итоговая сложность $O(\log(n))$.

\section*{Task 6}
Заметим, что функция $f(x) = \sum_{i=0}^n a_i x^i$ строго возрастающая. 
Так же, так как $f(x) \geq x$. \\
Вычислить $f(x)$ смтоит $O(n)$ операций. 
Будем поддерживать $x^k$ и $\sum_{k=0}^k a_i x^k$, переход к $k+1$ стоит $O(1)$ 
(добножить $x^k$ на $x$, затем на $a_0$ и прибавить к сумме).
Проверим, что $f(1) \leq y$, иначе решений нет. \\
Далее, пусть $l = 1, r = y$, тогда $f(1) \leq y \leq f(y)$.\\
Посчитаем $f\left( \frac{l + r}{2} \right)$. \\
\begin{itemize}
    \item $f\left( \frac{l + r}{2} \right) = y$, тогда решение найдено
    \item $f\left( \frac{l + r}{2} \right) < y$, тогда по монотонности решений на $l, \frac{l + r}{2}$ нет. 
    Обновим $l = \frac{l + r}{2}$.
    \item $f\left( \frac{l + r}{2} \right) > y$, тогда по монотонности решений на $\frac{l + r}{2}, r$ нет. 
    Обновим $r = \frac{l + r}{2}$.
    \item $r == l$. Решений нет.
\end{itemize}
Так как мы каждый раз уполовиниваем отрезок поиска, то максимальная глубина будет $O(\log(y))$. 
Каждая итерация стоит $O(n)$. Итоговая сложность $O(n \log(y))$.


\end{document}