\documentclass[12pt]{exam}
\usepackage{amsthm}
\usepackage{libertine}
\usepackage[utf8]{inputenc}
\usepackage[margin=1in]{geometry}
\usepackage{amsmath,amssymb}
\usepackage{multicol}
\usepackage[shortlabels]{enumitem}
\usepackage{siunitx}
\usepackage{cancel}
\usepackage{graphicx}
\usepackage{pgfplots}
\usepackage{listings}
\usepackage{tikz}
\usepackage{setspace}
\usepackage{tikz}

\usepackage[T2A]{fontenc}
\usepackage[utf8]{inputenc}
\usepackage[russian]{babel}


\pgfplotsset{width=10cm,compat=1.9}
\usepgfplotslibrary{external}

\newcommand{\class}{AI Masters Algorithms} 
\newcommand{\examnum}{Homework 6} 
\newcommand{\examdate}{\today} 
\newcommand{\timelimit}{}
\newcommand{\pluseq}{\mathrel{+}=}
\newcommand{\minuseq}{\mathrel{-}=}
\doublespacing


\begin{document}
\pagestyle{plain}
\thispagestyle{empty}

\noindent
\begin{tabular*}{\textwidth}{l @{\extracolsep{\fill}} r @{\extracolsep{6pt}} l}
\textbf{\class} & \textbf{Name:} & \textit{Денис Грачев}\\
\textbf{\examnum} &&\\
\textbf{\examdate} &&\\
\end{tabular*}\\
\rule[2ex]{\textwidth}{2pt}
% ---
\section*{Task 1}
Обозначим очереди как $a, b$.
Будем поддерживать следующий инвариант:
\begin{itemize}
    \item После каждой операции $b$ - пустая.
    \item Элементы в $a$ находятся в том же порядке что мы их добавляли.
\end{itemize}
Тогда операции push(x) и pop(x) устроены следующим образом.
\begin{figure}[h]
    {\tt push(x)}  \{ 
     
        \hspace{4mm} {\tt} a.add(x) 

    \} 

    \hfill \break

    {\tt pop()} \{
        
        \hspace{4mm} {\tt} while not a.empty() \{ 

            \hspace{8mm} {\tt} x = a.extract() 

            \hspace{8mm} {\tt} if not a.empty() \{ 

                \hspace{12mm} {\tt} b.add(x) 

            \hspace{8mm} \} 

        \hspace{4mm}\} 

        \hspace{4mm} {\tt} while not b.empty() \{ 

            \hspace{8mm} {\tt} a.add(b.extract()) 

        \hspace{4mm}\} 

        \hspace{4mm} {\tt} return x
    
    \}
\end{figure}

То есть при добавлении элемента, мы добавляем его в конец $a$ и инвариант сохраняется.
При доставании элемента мы перекладываем все кроме последнего элемента в $b$ и 
затем обратно в том же порядке в $a$, то есть инвариант сохраняется и возвращаем мы последний добавленный элемент.

\section*{Task 2}
Аналогично пронумеруем все вершины.
Докажем тогда, что вершина с номером $x$ имеет детей с номерами $3x - 1, 3x, 3x + 1$.
Соотвественно родитель вершины $x$ имеет номер $\lfloor \frac{x + 1}{3} \rfloor$.

\begin{tikzpicture}
    \node {1}
        child { node {2}
            child {node {5}}
            child {node {6}}
            child {node {7}}
        }
        child {node {3}}
        child {node {4}}
        ;
\end{tikzpicture}\\
Пусть это верно для слоя $h$. 
Тогда самый средний лист этого слоя имеет индекс $3^{h - 1}$. 
Всего листов на этом слое $3^{h - 1}$, 
соотвественно самый левый лист имеет индекс $l_h = 3^{h - 1} - \frac{3^{h - 1} - 1}{2}$, 
самый правый лист имеет индекс $3^{h - 1} + \frac{3^{h - 1} - 1}{2}$. \\
Тогда самый левый индекс слоя $h + 1$ имеет номер 
\begin{align*}
    3^{h - 1} + \frac{3^{h - 1} - 1}{2} + 1 
        &= \frac{2 * 3^{h - 1} + 3^{h - 1} + 1}{2} \\
        &= \frac{3^h + 1}{2} \\
        &= \frac{2 * 3^h - 3^h + 3 - 2}{2} \\
        &= 3^h - \frac{3^h - 3}{2} - 1 \\
        &= \left( 3^{h - 1} - \frac{3^{h - 1} - 1}{2} \right) 3 - 1\\
        &= l_h * 3 - 1
\end{align*}
Следовательно для самого левого листа слоя $h$ и его левого ребенка формула работает.
Очевидно что для остальных его детей формула так же работает, 
то есть они имеют индексы $3 l_h -1, 3 l_h, 3 l_h + 1$. 
Тогда формула работает и для следующей вершины слоя $h$ и тд, так как она имеет индекс 
$l_h + 1$, а ее дети $3l_h + 2, 3l_h + 3, 3l_h + 4 = 3(l_h + 1) - 1, 3(l_h + 1), 3(l_h + 1) + 1$.\\
База индукции видна на картинке. 

\section*{Task 3} 
\subsection*{y - предок x}
Пусть y не предок x. Тогда путь соединяющий x и y проходит через корень (обозначим корень r). 
Так как дерево бинарное, то они x и y по разным сторонам от r. \\
Таким образом 
\begin{itemize}
    \item либо x < r < y. Не может быть, так как y следующий по возрастанию от x.
    \item либо x > r > y. Не может быть, так как x < y.
\end{itemize}
Следовательно y - предок x.

\subsection*{y самый нижний, чей левый дочерний узел так же является предком x или самим x.}
Так как y - предок х и y > x, то x находится в левом от y поддереве. 
Следовательно левый ребенок х является либо предком x либо сам является x.\\
Пусть z обладает тем же свойством но находится ниже y. 
Тогда т.к. y - предок x и z предок x, но ниже, то y - предок z и по доказанному ранее 
z находится в левом поддерве y. Тогда x < z < y, противоречье.
Следовательно y самый низкий элемент обладающий таким свойством.

\section*{Task 4}
\subsection*{a предок b либо b предок a}
Рассмотрим вершину a. 
Докажем, что либо a предком b, либо b является предком a.
Аналогично предыдущей задаче, если они находятся по разные стороны от корня, то 
либо a < r < b, либо a > r > b, оба варианта противоречивы.
\subsection*{a предок b}
Пусть a предок b. 
Так как a < b, то b находится в правом поддереве а. 
Обозначим левого и правого ребенка b как l и r. 
Тогда a < l < b < r. Противоречье.
\subsection*{b предок a}
Пусть b предок a.
Так как a < b, то a находится в левом поддереве b. 
Пусть у a есть правый ребенок r, тогда a < r < b. 
Противоречье, следовательно правого ребенка у нее нет.

\subsection*{Для с}
Аналогично для с 
\begin{itemize}
    \item либо c предок b, либо b предок c
    \item c не может быть предком b
    \item у c не может быть левого ребенка
\end{itemize}

\section*{Task 5}
\subsection*{1}
Каждому множеству $A$ необходимо сопоставить уникальную строку длины $w$.
Пусть у разных множеств одинаковая строка. 
Тогда возьмем элемент который принадлежит одному мноежству но не принадлежит другому. 
В обоих случаях мы получим одинаковый ответ от нашей структуры, соответсвенно один будет неверный. \\
Мощность множества подмножеств длины $k$ равна ${n \choose k}$. \\
Мощность множества различных строк длины $w$ равна $2^w$.\\
Следовательно $2^w \geq {n \choose k}$. \\
Так же мы должны за $t$ запросов определять лежит ли элемент в $A$. \\
Количество возможных пар (вопрос, ответ) длины $t$ равно $2 ^ { {w \choose t} }$.



\end{document}