\documentclass[12pt]{exam}
\usepackage{amsthm}
\usepackage{libertine}
\usepackage[utf8]{inputenc}
\usepackage[margin=1in]{geometry}
\usepackage{amsmath,amssymb}
\usepackage{multicol}
\usepackage[shortlabels]{enumitem}
\usepackage{siunitx}
\usepackage{cancel}
\usepackage{graphicx}
\usepackage{pgfplots}
\usepackage{listings}
\usepackage{tikz}
\usepackage{setspace}

\usepackage[T2A]{fontenc}
\usepackage[utf8]{inputenc}
\usepackage[russian]{babel}


\pgfplotsset{width=10cm,compat=1.9}
\usepgfplotslibrary{external}
\tikzexternalize

\newcommand{\class}{AI Masters Algorithms} 
\newcommand{\examnum}{Homework 4} 
\newcommand{\examdate}{\today} 
\newcommand{\timelimit}{}
\doublespacing


\begin{document}
\pagestyle{plain}
\thispagestyle{empty}

\noindent
\begin{tabular*}{\textwidth}{l @{\extracolsep{\fill}} r @{\extracolsep{6pt}} l}
\textbf{\class} & \textbf{Name:} & \textit{Денис Грачев}\\
\textbf{\examnum} &&\\
\textbf{\examdate} &&\\
\end{tabular*}\\
\rule[2ex]{\textwidth}{2pt}
% ---

\section*{Task 1}
Очевидно что отсортированный массив будет выглядеть как 
$0, 0 \ldots 0, 1, 1, \ldots 1$, причем количество едениц и нулей останется прежним.
\begin{enumerate}
    \item Посчитать количество 1. Занимаем $\Theta(n)$, обозначим количество едениц $k$.
    \item Вписать в массив $n-k$ нулей затем $k$ едениц. Занимает $\Theta(n)$
\end{enumerate}
Очевидно что полученный массив соответсвует сортировке изначального массивва.

\section*{Task 2}
Так как отрезки строго вложенные, то порядок верно следующее 
$l_i < l_j \Leftrightarrow [l_j, r_j] \subset [l_i, r_i]$, так же $\forall i \neq j: l_i \neq l_j$. \\
Будем двигать точку $x$ с самого левого края к самому правому. 
Легко видеть что сначала она пересечет $n$ левых краев, затем $n$ правых краев. 
Каждый раз при пересечении левого края количество покрываемых ею отрезков увеличивается на 1, 
каждый раз при пересечении правого отрезка количество покрываемых ею отрезков уменьшается на 1.
Отсюда легко видеть что точка будет покрыта ровно $\frac{2n}{3}$ отрезками если она находится 
между $\frac{2n}{3}$ и $\frac{2n}{3} + 1$ левым или правым отрезком 
(левые надо считать слева, правые справа. Понятно что это будут края одних и тех же отрезков). \\
Будем сравнивать отрезки по левому краю. Найдем $\frac{2n}{3}$ и $\frac{2n}{3} + 1$ порядковые статистики, 
обозначим их как $k$ и $m$. Это занимает сложность $O(n)$. 
Искомые точки находятся в отрезках 
$[l_k, l_m]$ и $[r_m, l_k]$.

\section*{Task 3}
\subsection*{1}
На каждом новом запуске $\mathrm{quicksort}$ длина входа уменьшается хотя бы на 1. 
Таким образом она не может быть больше $n$. 
Длина достигает $n$ на массиве $1, 2, 3, \ldots n$, так как каждый раз последний элемент берется как $\mathrm{pivot}$ 
и $\mathrm{partition}$ возвращает индекс последнего элемента.

\subsection*{2}
Каждый раз при запуске будем находить медиану текущего массива и менять ее местами с последним элементом.\\
Тогда вход будет каждый раз делиться на 2 и глубина будет $\Theta(\log(n))$.

\section*{Task 4}
Пусть $x_1 \leq x_2 \leq \ldots x_n$.\\
Наиболее наглядный способ - нарисовать график функции $f(s) = \sum_{i = 1}^{2n + 1} | x_i - s |$.
Функция непрерывная, докажем что $x_{n+1}$ - минимум. 

 
\end{document}