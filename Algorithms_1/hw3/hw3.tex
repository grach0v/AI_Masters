\documentclass[12pt]{exam}
\usepackage{amsthm}
\usepackage{libertine}
\usepackage[utf8]{inputenc}
\usepackage[margin=1in]{geometry}
\usepackage{amsmath,amssymb}
\usepackage{multicol}
\usepackage[shortlabels]{enumitem}
\usepackage{siunitx}
\usepackage{cancel}
\usepackage{graphicx}
\usepackage{pgfplots}
\usepackage{listings}
\usepackage{tikz}
\usepackage{setspace}

\usepackage[T2A]{fontenc}
\usepackage[utf8]{inputenc}
\usepackage[russian]{babel}
\usepackage{mathtools}
\DeclarePairedDelimiter\ceil{\lceil}{\rceil}
\DeclarePairedDelimiter\floor{\lfloor}{\rfloor}

\pgfplotsset{width=10cm,compat=1.9}
\usepgfplotslibrary{external}
\tikzexternalize

\newcommand{\pluseq}{\mathrel{+}=}
\newcommand{\minuseq}{\mathrel{-}=}

\newcommand{\class}{AI Masters Algorithms} 
\newcommand{\examnum}{Homework 3} 
\newcommand{\examdate}{\today} 
\newcommand{\timelimit}{}
\doublespacing

\begin{document}
\pagestyle{plain}
\thispagestyle{empty}

\noindent
\begin{tabular*}{\textwidth}{l @{\extracolsep{\fill}} r @{\extracolsep{6pt}} l}
\textbf{\class} & \textbf{Name:} & \textit{Денис Грачев}\\
\textbf{\examnum} &&\\
\textbf{\examdate} &&\\
\end{tabular*}\\
\rule[2ex]{\textwidth}{2pt}
% ---

\section*{Task 1}
Запишем рекурентную формулу (для $n \leq 2022$ считаем $f(n)$ константой не превышающей 2022)
$$ f(n) = 3f(n / 4) + c $$
$$ d = \log_4(3) < \frac{1}{2} $$
Первый случай  
$$ c = O(n^{d - \varepsilon}) \: \textit{где} \: \varepsilon = d - \frac{1}{2}$$ 
Тогда из теоремы о рекурсии $a = 3, b = 4, f(n) = c$ следует что
$f(n) = \Theta (n ^ {\log_4(3)})$ 

\section*{Task 2}
\subsection*{a}
$$ T(n) = 36 T\left( \floor*{\frac{n}{6}} \right) + n^2$$
Применим теорему о рекурсии, где $a = 36, b = 6, f(n) = n^2$, \\
тогда $d = \log_6(36) = 2, \: f(n) = \Theta(n^d)$. \\
Следовательно $T(n) = \Theta(n^2 \log(n))$

\subsection*{b}
$$ T(n) = 3 T \left( \floor*{\frac{n}{3}} \right) + n^2$$
Применим теорему о рекурсии, где $a = 3, b = 3, f(n) = n^2$, \\
тогда $d = \log_3(3) = 1,
\: f(n) = n^2 = \Omega(n^{d + 0.1 = 1.1}), 
\: a f \left( \frac{n}{b} \right) = 3 \frac{n^2}{9} = \frac{n^2}{3} \leq n^2 = f(n)$ \\
По третьему случаю $T(n) = \Theta(f(n)) = \Theta(n^2)$

\subsection*{c}
$$T(n) = 4 T \left( \floor*{ \frac{n}{2}} \right) + \floor*{\frac{n}{\log(n)}}$$
Применим теорему о рекурсии, где $a = 4, b = 2, f(n) = \frac{n}{\log(n)}$, \\
тогда $d = \log_2(4) = 2, 
\: f(n) = \frac{n}{\log (n)} \leq n = O(n^{d - 1 = 1})$. \\
По 1 случаю $T(n) = \Theta(n^d) = \Theta(n^2)$.

\section*{Task 3}
\subsection*{1}
$$ T(n) = n T \left( \frac{n}{2} \right) + cn$$
Высота дерева будет $h = \log_2(n)$. \\
Посчитаем сколько операций на каждом уровне рекурсии.\\
\begin{center}
\begin{tabular}{c | c | c | c | c}
    глубина дерева & аргумент листа &количество листьев & операций в листе & операций всего \\
    \hline
    1 & $n$ & 1 & $cn$ & $cn$\\
    2 & $\frac{n}{2}$ & $n$ & $c \frac{n}{2}$ & $n c\frac{n}{2} = c\frac{n^2}{2}$ \\
    3 & $\frac{n}{4}$ & $\frac{n}{2} n = \frac{n^2}{2}$ & $c \frac{n}{4}$ & $\frac{n^2}{2} c\frac{n}{4} = c\frac{n^3}{8}$ \\
    4 & $\frac{n}{8}$ & $\frac{n}{4} \frac{n^2}{2} = \frac{n^3}{8}$ & $c \frac{n}{8}$ & $\frac{n^3}{8} c\frac{n}{8} = c\frac{n^4}{64}$ \\
    5 & $\frac{n}{16}$ & $\frac{n}{8} \frac{n^3}{8} = \frac{n^4}{64}$ & $c \frac{n}{16}$ & $\frac{n^4}{64} c\frac{n}{16} = c\frac{n^5}{1024}$ \\
    $\ldots$ & $\ldots$ & $\ldots$ & $\ldots$ & $\ldots$ \\
    
    k + 1 & 
    $\frac{n}{2^k} = 2^{h - k}$ & 
    $\frac{n^k}{2^{\frac{k(k-1)}{2}}} = 2^{hk - \frac{k(k-1)}{2}}$ & 
    $c\frac{n}{2^k} = c 2 ^ {h - k}$ & 
    $c 2 ^ {hk - \frac{k(k-1)}{2} + h - k}$ \\ 
    $\ldots$ & $\ldots$ & $\ldots$ & $\ldots$ & $\ldots$ \\

    h & 1 & $2 ^ {h(h-1) - \frac{(h-1)(h-2)}{2}}$ & $c$ & $c2 ^ {h(h-1) - \frac{(h-1)(h-2)}{2}}$  
    
\end{tabular}
\end{center}

\textit{В формулах в табличке выше могли потеряться +- 1, в формуле ниже все работает}\\
\begin{align*}
    T(h) &= \sum_{k = 1}^h c 2 ^ {h(k - 1) - \frac{(k - 1)(k - 2)}{2} + h - k + 1}  \\
         &= c \sum_{k = 1}^h 2 ^ {hk - \frac{k^2 + k}{2}} \\
         &\leq c \sum_{k = 1}^h 2 ^ {h k} \\
         &= c  \left( \frac{2^{h^2 + h} - 2^{h}}{2^{h} - 1} \right)  \\
         &= c \left( \Theta \left( 2^{h^2} \right)\right) \\
         &= \Theta \left(  2^{h^2} \right) = \Theta ({2^{h}} ^ { h}) = \Theta (n^{\log(n)})
\end{align*}

Таким образом $T(n) = O(n^{\log(n)})$. \\
Рассмотрим $hk - \frac{k^2 + k}{2}$ как параболу относительно $k$. \\
Так как ее корни $0, 2h -1$, то максимальное минимальное значение достигается при $k=h$.

\begin{align*}
    T(h) &= \sum_{k = 1}^h c 2 ^ {h(k - 1) - \frac{(k - 1)(k - 2)}{2} + h - k + 1}  \\
         &= c \sum_{k = 1}^h 2 ^ {hk - \frac{k^2 + k}{2}} \\
         &\geq c \sum_{k = 1}^h 2 ^ {h^2 - \frac{h^2 + h}{2}} \\
         &= c h 2^{\frac{h^2 - h}{2}} \\
         &= \Theta(\log(n) n ^ {\frac{\log(n) - 1}{2}})
\end{align*}
Таким образом $T(n) = \Omega(\log(n) n ^ {\frac{\log(n) - 1}{2}})$. \\
Точнее оценить пока не удалось.

\section*{Task 4}
Пусть битовая длина максимального из чисел $n$. 
Тогда с помощью алгоритма умножения Карацубы можно посчитать $ab$ за $O(n^{log_2(3)})$.
Опишем алгоритм деления в двоичной системе.
\begin{itemize}
    \item умножим делитель на такую степень 2, чтобы длины делимого и делителя совпали
    \item вычтем из делимого полученное число, запомним результат
\end{itemize}
Повторять пока делимое больше делителя. Таким образом мы каждый раз отрезаем от делителя 1 бит а вычитание стоит $O(n)$. 
Итоговая сложность деления $O(n^2)$. \\
Чтобы посчитать НОД нам необходимо сделать $O(n)$ делений. Итоговая сложность $O(n^3)$.\\
$\mathrm{lcm}(a, b) = \frac{ab}{\gcd(a, b)}$. Битовая длина $ab < 2n$, соответственно сложность деления $O((2n)^2) = O(4n^2) = O(n^2)$. \\
Итоговая сложность $O(n^{\log_2(3)} + n^3 + n^2) = O(n^3)$. 

\section*{Task 5}
Для того чтобы посчитать количество инверсий, заведем счетчик который будет обновлять во время слияния.\\
При слиянии подмассивов $a, b$ ($a$ - левый, $b$ - правый), 
если $a_i > b_j$ это значит, что $\forall k \leq j: a_i > b_k$ так как они отсортированы, 
при этом изначально $b_k$ были правее $a_i$, соотвественно у нас было $j$ инверсий, 
добавим к счетчику $j$.  \\
Почему все инверсии посчитаются и ровно 1 раз. 
Когда мы переставляем $a_i$ элемент перед $b_j$ то устраняем ровно $j$ инверсий. \\
В отсортированном массиве инверсий нет. Значит все инверсии посчитаны.

\section*{Task 6}
$ab = \frac{(a + b)^2 - a^2 - b^2}{2}$. \\
Для вычисления этого числа необходимо произвести
\begin{itemize}
    \item Сложение $O(n)$
    \item Возведение в квадрат $O(n)$
    \item Возведение в квадрат $O(n)$
    \item Возведение в квадрат $O(n)$
    \item Вычитание $O(n)$
    \item Вычитание $O(n)$
    \item Деление на 2 $O(1)$
\end{itemize}
Итоговая сложность $O(n)$.

\section*{Task 7}
\subsection*{a}
$T(n) = T(\alpha n) + T((1 - \alpha) n) + \Theta(n)$\\
Количество операций на каждом уровне $\Theta(n)$. \\
Минимальная глубина дерева $-\log_{\alpha} (n)$, максимальная $-\log_{1 - \alpha}(n)$.\\
Значит количество операций зажато между $-\log_{\alpha} (n) n \leq T(n) \leq -\log_{1 - \alpha}(n) n$.\\
Значит $T(n) = \Theta(n \log(n))$.

\subsection*{b}
$T(n) = T(\frac{n}{2}) + 2T(\frac{n}{4}) + \Theta(n)$\\
Количество оперций на каждом уровне рекурсии $n$. \\
глубина рекурсии между $\log_4(n)$ и $\log_2(n)$. \\
Таким образом $T(n) = \Theta(n \log(n))$.\\

\subsection*{c}
$T(n) = 27 T(\frac{n}{3}) + \frac{n^3}{\log^2(n)}$ \\
рассмотрим количество операций на следующем уровне рекурсии 
$27 \frac{n^3 / 3^3}{\log^2(n / 3)} = \frac{n^3}{\log^2(n / 3)}$.\\
\begin{align*}
    T(n) &= \sum_{k=0}^{\log_3(n)} \frac{n^3}{\log^2(n / 3^k)} \\
         &= n^3 \sum_{k=0}^{\log_3(n)} \frac{1}{\log^2(n / 3^k)} \\
         &= n^3 \sum_{k=0}^{\log_3(n)} \frac{1}{(\log(n) - k \log(3))^2} = \Theta\left(\frac{n^3}{\log^2(n)}\right)
\end{align*}


\section*{Task 8}
Пусть $n$ нечетное. 
Существует три варианта расположения $[a_i, a_{i + 1} \ldots a_{j}]$:
\begin{itemize}
    \item $j < \frac{n + 1}{2}$ - полностью лежит до середины.
    \item $i > \frac{n + 1}{2}$ - полностью лежит после середины.
    \item $i \leq \frac{n + 1}{2} \leq j$ - пересекает середину.
\end{itemize}
Найдем максимумы 1 и 2 случая с помощью рекурсивного вызова функции для левой и правой половинок.\\
Решим задачу за $\Theta(n)$ для 3 случая. \\
Заведем два индекса $\mathrm{lhs} = \frac{n + 1}{2} = \mathrm{rhs}$ 
и два минимума $\min_e = a_{\frac{n + 1}{2}} = \max_s$.
Будем изменять их по следующему принципу:
\begin{tabbing}
    Если \= $a_{\mathrm{lhs} - 1} > a_{\mathrm{rhs} + 1}$: \\
    \> $\mathrm{lhs} \minuseq 1$\\
    \> $\min_e = \min(\min_e, a_{\mathrm{lhs}}) $\\
    Иначе:\\
    \> $\mathrm{rhs} \pluseq 1$\\
    \> $\min_e = \min(\min_e, a_{\mathrm{rhs}}) $\\
    
    $\max_s = \max(\max_s, (rhs - lhs + 1) \min_e)$
\end{tabbing}
Если какой то индекс доходит до края, перестаем его изменять. \\
Такое действие занимает $\Theta(n)$ по сложности, 
так как каждый раз какой то из индексов отъезжает от середины и расстояние между ними не может быть больше $n$.
Докажем, что в конце переменная $\max_s$ будет хранить максимальное возможное искомое значение среди подмассивов проходящих через середину.
Предположим противное, тогда есть подмассив края которого 
$\mathrm{lhs}_{w}$, $\mathrm{rhs}_w$ и минимальный элемент $\min_w$
для которого искомое значение больше того что мы получили. \\
Рассмотрим последний момент когда наш расширяющийся подмассив 
был внутри $\mathrm{lhs}_w, \: \mathrm{rhs}_w$. 
Они не могли идеально совпадать, так как тогда наилучшее значение было бы записано в $\max_s$ и никогда более бы не изменилось.\\
Значит один из краев у них совпадает и в следующий момент этот край выйдет за пределы наилучшего подмассива.\\
Без ограничения общности можем считать что это левый край, то есть $\mathrm{lhs} = \mathrm{lhs}_w$, $\mathrm{rhs} < \mathrm{rhs}_w$ и 
в следующий момент мы сделаем $\mathrm{lhs} \minuseq 1$, следовательно $a_{\mathrm{lhs} - 1} > a_{\mathrm{rhs} + 1}$.\\
$\min_w \leq a_{\mathrm{rhs} + 1}$ так как $a_{\mathrm{rhs} + 1}$ находится внутри $\mathrm{lhs}_w, \mathrm{rhs}_2$.\\
$\min_w \leq a_{\mathrm{rhs} + 1} \leq a_{\mathrm{lhs} - 1}$. \\
Значит размер нашего идеального подмассива можно увеличить и значение не наилучшее, противоречье.

Пусть сложность это $T(n) = 2T(n/2) + \Theta(n)$. \\
Аналогично сортировке из лекции $T(n) = \Theta(n \log(n))$

Для того чтобы найти не только наилучшее значение но и индексы подмассива, будем вместе с наилучшим значением хранить науилчшие $\mathrm{lhs}$ и $\mathrm{rhs}$.
Случай когда размер переданного отрекзка равен 1 тривиален.
\end{document}




















