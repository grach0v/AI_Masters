\documentclass[12pt]{exam}
\usepackage{amsthm}
\usepackage{libertine}
\usepackage[utf8]{inputenc}
\usepackage[margin=1in]{geometry}
\usepackage{amsmath,amssymb}
\usepackage{multicol}
\usepackage[shortlabels]{enumitem}
\usepackage{siunitx}
\usepackage{cancel}
\usepackage{graphicx}
\usepackage{pgfplots}
\usepackage{listings}
\usepackage{tikz}
\usepackage{setspace}


\usepackage[T2A]{fontenc}
\usepackage[utf8]{inputenc}
\usepackage[russian]{babel}
\usepackage{mathtools}
\DeclarePairedDelimiter\ceil{\lceil}{\rceil}
\DeclarePairedDelimiter\floor{\lfloor}{\rfloor}

\pgfplotsset{width=10cm,compat=1.9}
\usepgfplotslibrary{external}
\tikzexternalize

\newcommand{\class}{AI Masters Algorithms} 
\newcommand{\examnum}{Homework 3} 
\newcommand{\examdate}{\today} 
\newcommand{\timelimit}{}
\doublespacing

\begin{document}
\pagestyle{plain}
\thispagestyle{empty}

\noindent
\begin{tabular*}{\textwidth}{l @{\extracolsep{\fill}} r @{\extracolsep{6pt}} l}
\textbf{\class} & \textbf{Name:} & \textit{Денис Грачев}\\
\textbf{\examnum} &&\\
\textbf{\examdate} &&\\
\end{tabular*}\\
\rule[2ex]{\textwidth}{2pt}
% ---

\section*{Task 1}
Запишем рекурентную формулу (для $n \leq 2022$ считаем $f(n)$ константой не превышающей 2022)
$$ f(n) = 3f(n / 4) + c $$
$$ d = \log_4(3) < \frac{1}{2} $$
Первый случай  
$$ c = O(n^{d - \varepsilon}) \: \textit{где} \: \varepsilon = d - \frac{1}{2}$$ 
Тогда из теоремы о рекурсии $a = 3, b = 4, f(n) = c$ следует что
$f(n) = \Theta (n ^ {\log_4(3)})$ 

\section*{Task 2}
\subsection*{a}
$$ T(n) = 36 T\left( \floor*{\frac{n}{6}} \right) + n^2$$
Применим теорему о рекурсии, где $a = 36, b = 6, f(n) = n^2$, \\
тогда $d = \log_6(36) = 2, \: f(n) = \Theta(n^d)$. \\
Следовательно $T(n) = \Theta(n^2 \log(n))$

\subsection*{b}
$$ T(n) = 3 T \left( \floor*{\frac{n}{3}} \right) + n^2$$
Применим теорему о рекурсии, где $a = 3, b = 3, f(n) = n^2$, \\
тогда $d = \log_3(3) = 1,
\: f(n) = n^2 = \Omega(n^{d + 0.1 = 1.1}), 
\: a f \left( \frac{n}{b} \right) = 3 \frac{n^2}{9} = \frac{n^2}{3} \leq n^2 = f(n)$ \\
По третьему случаю $T(n) = \Theta(f(n)) = \Theta(n^2)$

\subsection*{c}
$$T(n) = 4 T \left( \floor*{ \frac{n}{2}} \right) + \floor*{\frac{n}{\log(n)}}$$
Применим теорему о рекурсии, где $a = 4, b = 2, f(n) = \frac{n}{\log(n)}$, \\
тогда $d = \log_2(4) = 2, 
\: f(n) = \frac{n}{\log (n)} \leq n = O(n^{d - 1 = 1})$. \\
По 1 случаю $T(n) = \Theta(n^d) = \Theta(n^2)$.

\newpage
\section*{Task 3}
\subsection*{1}
$$ T(n) = n T \left( \frac{n}{2} \right) + cn$$
Высота дерева будет $h = \log_2(n)$. \\
Посчитаем сколько операций на каждом уровне рекурсии.\\
\begin{center}
\begin{tabular}{c | c | c | c | c}
    глубина дерева & аргумент листа &количество листьев & операций в листе & операций всего \\
    \hline
    1 & $n$ & 1 & $cn$ & $cn$\\
    2 & $\frac{n}{2}$ & $n$ & $c \frac{n}{2}$ & $n c\frac{n}{2} = c\frac{n^2}{2}$ \\
    3 & $\frac{n}{4}$ & $\frac{n}{2} n = \frac{n^2}{2}$ & $c \frac{n}{4}$ & $\frac{n^2}{2} c\frac{n}{4} = c\frac{n^3}{8}$ \\
    4 & $\frac{n}{8}$ & $\frac{n}{4} \frac{n^2}{2} = \frac{n^3}{8}$ & $c \frac{n}{8}$ & $\frac{n^3}{8} c\frac{n}{8} = c\frac{n^4}{64}$ \\
    5 & $\frac{n}{16}$ & $\frac{n}{8} \frac{n^3}{8} = \frac{n^4}{64}$ & $c \frac{n}{16}$ & $\frac{n^4}{64} c\frac{n}{16} = c\frac{n^5}{1024}$ \\
    $\ldots$ & $\ldots$ & $\ldots$ & $\ldots$ & $\ldots$ \\
    
    k + 1 & 
    $\frac{n}{2^k} = 2^{h - k}$ & 
    $\frac{n^k}{2^{\frac{k(k-1)}{2}}} = 2^{hk - \frac{k(k-1)}{2}}$ & 
    $c\frac{n}{2^k} = c 2 ^ {h - k}$ & 
    $c 2 ^ {hk - \frac{k(k-1)}{2} + h - k}$ \\ 
    $\ldots$ & $\ldots$ & $\ldots$ & $\ldots$ & $\ldots$ \\

    h & 1 & $2 ^ {h(h-1) - \frac{(h-1)(h-2)}{2}}$ & $c$ & $c2 ^ {h(h-1) - \frac{(h-1)(h-2)}{2}}$  
    
\end{tabular}
\end{center}

\textit{В формулах в табличке выше могли потеряться +- 1, в формуле ниже все работает}\\
\begin{align*}
    T(n) &= \sum_{k = 1}^h c 2 ^ {h(k - 1) - \frac{(k - 1)(k - 2)}{2} + h - k + 1}  \\
         &= c \sum_{k = 1}^h 2 ^ {hk - \frac{k^2 + k}{2}}
\end{align*}
Получившаяся сумма не является геометрической прогрессией и явно оценить ее сложность не получилось. \\
Однако можно оценить сложность $\log (T(n))$. \\
\begin{align*}
\log (T(n)) &= \log (c \sum_{k = 1}^h 2 ^ {hk - \frac{k^2 + k}{2}}) \qquad \textit{т.к.} \: k^2 \leq kh \\
            &= c' \log \left( \sum_{k = 1}^h \Theta(2 ^ {c_1 h k}) \right) \\
            &= c' \log \left( \Theta \left( \frac{2^{c_1 h^2 + c_1h} - 2^{c_1h}}{2^{c_1h} - 1} \right) \right) \\
            &= c' \log \left( \Theta \left( 2^{c_1 h^2} \right)\right) \\
            &= \Theta \left( \log( 2^{c1 h^2} ) \right) \\
            &= \Theta (h^2)
\end{align*}

\section*{Task 4}


\end{document}




















