\documentclass[12pt]{exam}
\usepackage{amsthm}
\usepackage{libertine}
\usepackage[utf8]{inputenc}
\usepackage[margin=1in]{geometry}
\usepackage{amsmath,amssymb}
\usepackage{multicol}
\usepackage[shortlabels]{enumitem}
\usepackage{siunitx}
\usepackage{cancel}
\usepackage{graphicx}
\usepackage{pgfplots}
\usepackage{listings}
\usepackage{tikz}
\usepackage{setspace}
\usepackage{centernot}

\usepackage[T2A]{fontenc}
\usepackage[utf8]{inputenc}
\usepackage[russian]{babel}


\pgfplotsset{width=10cm,compat=1.9}
\usepgfplotslibrary{external}
\tikzexternalize

\newcommand{\class}{AI Masters Algorithms} 
\newcommand{\examnum}{HomeWork 2} 
\newcommand{\examdate}{\today} 
\newcommand{\timelimit}{}
\doublespacing


\begin{document}
\pagestyle{plain}
\thispagestyle{empty}

\noindent
\begin{tabular*}{\textwidth}{l @{\extracolsep{\fill}} r @{\extracolsep{6pt}} l}
\textbf{\class} & \textbf{Name:} & \textit{Денис Грачев}\\ %Your name here instead, obviously 
\textbf{\examnum} &&\\
\textbf{\examdate} &&\\
\end{tabular*}\\
\rule[2ex]{\textwidth}{2pt}
% ---


\section*{Задача 1}
\subsection*{a}

$238x + 385y = 133$
\begin{center}
\begin{tabular}{c | c | c}
    $x$ & $y$ & $238 x + 385 y$ \\
    \hline 
    0 & 1 & 385 \\
    1 & 0 & 238 \\ 
    -1 & 1 & 147 \\
    2 & -1 & 91 \\ 
    -3 & 2 & 56 \\
    5 & -3 & 35 \\
    -8 & 5 & 21 \\
    13 & -8 & 14 \\
    -21 & 13 & 7 \\
    55 & -34 & 0

\end{tabular}
\end{center}
Следовательно \\
$238a + 385b = 0 \Leftrightarrow a, b = 55k, -34k$, \\
$\gcd (238, 385) = 7 = -21 * 238 + 13 * 385$.\\
$133 : 7 = 19 \Rightarrow (-21 * 19) * 238 + (13 * 19) * 385 = 133$ \\
$x = -21 * 19 = -399, y = 13 * 19 = 247$. \\ 
Пусть $238x' + 385y' = 133 
\Rightarrow 238(x - x') + 385(y - y') = 0 
\Rightarrow x', y' = x + 55k, y - 34k$. \\
Ответ: $x, y = -399 + 55k, 247 - 34k$, где $k \in \mathbb{Z}$.

\subsection*{б}
$143x + 121y = 52$
\begin{center}
\begin{tabular}{c | c | c}
    $x$ & $y$ & $143 x + 121 y$ \\
    \hline 
    1 & 0 & 143 \\
    0 & 1 & 121 \\
    1 & -1 & 22 \\
    -5 & 6 & 11 \\
    11 & -13 & 0 
    
\end{tabular}
\end{center}
Следовательно \\
$143 + 121b = 0 \Leftrightarrow a, b = 11k, -13k$, \\
$\gcd (143, 121) = 11 = -5 * 143 + 6 * 121$.\\
$ 52 \centernot\vdots 11$ Решения в целых числах не существует. 

\section*{Задача 2}
$68x + 85 \equiv 0 (\mod 561) \Leftrightarrow 
\exists N \in \mathbb{Z}: 68x + 85 = 561N \Leftrightarrow 
\exists N \in \mathbb{Z}: 68x + 561N = 476$. \\

\begin{center}
    \begin{tabular}{c | c | c}
        $x$ & $N$ & $68 x + 561 N$ \\
        \hline 
        0 & 1 & 561 \\ 
        1 & 0 & 68 \\ 
        -8 & 1 & 17 \\
        33 & -4 & 0 
        
    \end{tabular}
\end{center}
Следовательно \\
$68 a + 561 b = 0 \Leftrightarrow a, b = 33k, -4k$,\\
$\gcd(68, 561) = 17 = -8 * 68 + 1 * 561$.\\
$476 : 17 = 28 \Rightarrow (-8 * 28) * 68 + (1 * 68) * 561 = 476$\\
$x = (-8 * 28) = -224 \equiv 337 (\mod 561)$. \\
Аналогично задаче1, $x = 337 + 33k = 7 + 33k$.\\
Ответ $x = 7 + 33k, k \in \mathbb{Z}$

\section*{Задача 3}
Третий и больше столбики заполняется снизу вверх.\\
\textit{Я пытался выровнять знаки равенства, почему то выровнялись только mod}
\begin{align*}
    7^{13} &\mod 167 &= 7 * {7^{6}} ^ 2 &\mod 167 &= 7 * 81^2 &\mod 167 &= 7 * 48 \mod 167 = 2\\
    7^6    &\mod 167 &= {7^{3}}^2       &\mod 167 &= 9^2      &\mod 167 &= 81\\
    7^3    &\mod 167 &= 7 * 7^2         &\mod 167 &= 7 * 49   &\mod 167 &= 9 \\
    7^1    &\mod 167 &= 7               &\mod 167 &= 7        &\mod 167 &= 7             
\end{align*}

\section*{Задача 5}
\subsection*{1}
$T_1(n) = cn + T_1(n - 1)$\\
$T_1(n) = cn + c(n - 1) + T_1(n - 2)$\\
$T_1(n) = cn + c(n - 1) + c(n - 2) + \ldots + c*4 + T(3)$\\
$T_1(n) = c(n + (n - 1) + (n - 1) + \ldots 1) - c (3 + 2 + 1) + T(3)$\\
$T_1(n) = c\frac{n (n + 1)}{2} - 6c + 1$\\
Легко видеть что $T_1(n) = \Theta(n^2)$

\subsection*{2}
$T_2(n) = T_2(n - 1) + 4T_2(n - 3)$\\
С заменой $n = n - 3$, чтобы пропустить часть которая определяется не по формуле.
\begin{itemize}
    \item Легко видеть, что $T_2(n)$ монотонно возрастающая функция
    \item $T_2(n) \geq 4T_2(n - 3) \Rightarrow \frac{T_2(n)}{T_2(n-3)} \geq 4 \Rightarrow T_2(n) \geq {(4^{\frac{1}{3}})} ^ n \Rightarrow \log T_2(n) = \Omega(n)$.
    \item $T_2(n) \leq 5T_2(n - 1) \Rightarrow \frac{T_2(n)}{T_2(n-1)} \leq 5 \Rightarrow T_2(n) \leq 5^n \Rightarrow \log T_2(n) = O(n)$
\end{itemize}
Таким образом $\log T_2(n) = \Theta (n)$

\section*{6}
Пусть длина это бит в регистре $n$. \\
Сделаем функцию которая копирует $k$-ый бит и не меняет значения битов после $k$. \\
$r$ - регистр, $k$ - в соответсвии с описанием, $v$ - значение бита записанного в $k$.
\begin{lstlisting}
    SetK(r, k, v):
        if k == 1:
            r[1] = v
            return

        SetK(r, k-1, 1)
        for i in [k-2 ... 1]:
            SetK(r, i, 0)
        
        r[k] = v
        return 
\end{lstlisting}

\subsection*{Корректность}
Функция рекурсивная, аргумент k с каждым вызовом уменьшается. 
Есть условие остановки при k=1 поэтому функция конечная. \\
Докажем корректность по индукции:
\subsubsection*{База}
Для k=1 корректность очевидна
\subsubsection*{Переход}
Пусть функция корректна для $i = 1 \ldots k-1$, докажем корректность для $k$. \\
Тогда SetK(r, k-1, 1) выставит 1 в k-1 бит и не изменит биты после k. \\
SetK(r, i, 0) выставят 0 в i элемент и не изменит биты после i, таким образом все биты от 1 до k-1 будут 0.\\
Таким образом в k-1 будет стоять первая 1, и r[k] = v поставит необходимое значение в $k$-ый бит.\\

\subsection*{Алгоритм}
\begin{lstlisting}
    Copy(from, to, n):
        for i in [n ... 1]:
            SetK(to, i, from[i])
\end{lstlisting}
Корректность алгоритма легко следует из корректности функции SetK.

\subsection*{Сложность}
Обозначим сложность функции SetK(r, k, v) как $T(k)$. \\
Тогда $T(k) = 1 + \sum_{i = 1}^{k - 1} T(i)$ \\
Докажем что $T(k) = 2^{k - 1}$ по индукции. \\
База: для $k = 1$ верно.
Шаг индукции: \\
Пусть верно для $i = 1 \ldots k - 1$, тогда \\
$T(k) = 1 + T(1) + T(2) + \ldots T(k - 1) = 1 + 1 + 2 + 4 + \ldots 2^{k-2} = 2^{k-1}$\\
Итоговая сложность будет $T(1) + T(2) + \ldots T(n) = 2^n - 1$, т.е. $\Theta(2^n)$.


\end{document}