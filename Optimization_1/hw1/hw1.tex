\documentclass[12pt]{exam}
\usepackage{amsthm}
\usepackage{libertine}
\usepackage[utf8]{inputenc}
\usepackage[margin=1in]{geometry}
\usepackage{amsmath,amssymb}
\usepackage{multicol}
\usepackage[shortlabels]{enumitem}
\usepackage{siunitx}
\usepackage{cancel}
\usepackage{graphicx}
\usepackage{pgfplots}
\usepackage{listings}
\usepackage{tikz}
\usepackage{setspace}

\usepackage[T2A]{fontenc}
\usepackage[utf8]{inputenc}
\usepackage[russian]{babel}

\pgfplotsset{width=10cm,compat=1.9}
\usepgfplotslibrary{external}
\tikzexternalize

\newcommand{\class}{AI Masters Optimisatrion} 
\newcommand{\examnum}{Homework 1} 
\newcommand{\examdate}{\today} 
\newcommand{\timelimit}{}
\doublespacing


\begin{document}
\pagestyle{plain}
\thispagestyle{empty}

\noindent
\begin{tabular*}{\textwidth}{l @{\extracolsep{\fill}} r @{\extracolsep{6pt}} l}
\textbf{\class} & \textbf{Name:} & \textit{Денис Грачев}\\
\textbf{\examnum} &&\\
\textbf{\examdate} &&\\
\end{tabular*}\\
\rule[2ex]{\textwidth}{2pt}
% ---

\section*{Task 1}
\subsection*{$\Rightarrow$}
Пусть $V$ - выпуклое множество. 
Прямая $l$ так же выпклое множество.
Пересечение выпкулых множеств выпукло, следовательно $V \cap l$ - выпуклое.

\subsection*{$\Leftarrow$}
Пусть $V: \: \forall l: \: V \cap l - \textit{выпуклое} $. \\
Тогда $\forall x, y \in V$ рассмотрим $l: x, y \in l$. 
Тогда $x, y \in V \cap l$ следовательно, так как $V \cap l - \textit{выпуклое}$, 
то $[x, y] \in V \cap l \Rightarrow [x, y] \in V$, следовательно $V$ - выпуклое.

\section*{Task 2}
$\mathcal{C}=\left\{\mathbf{x} \in \mathbb{R}^n \mid \mathbf{x}^{\top} \mathbf{A} \mathbf{x}+\mathbf{b}^{\top} \mathbf{x}+c \leq 0\right\}$.\\
Для того чтобы доказать что $\mathcal{C}$ - выпуклое, 
достаточно доказать что $\forall x, y \in \mathcal{C}: \frac{x + y}{2} \in \mathcal{C}$ 
(для любой точки между $x, y$ можно построить сходяющуюся последовательность из середин,
так как $\mathcal{C}$ замкнутое, то и предел последовательности тоже будет лежать в $\mathcal{C}$). \\
Пусть $x, y \in \mathcal{C}$, тогда 
\begin{equation}
    \begin{cases}
        x^T A x + b^T x + c \leq 0 \qquad \textit{т.к.} \: x \in \mathcal{C} \\
        y^T A y + b^T y + c \leq 0 \qquad \textit{т.к.} \: y \in \mathcal{C} \\
        (x - y)^T A (x - y) \geq 0 \qquad \textit{т.к.} \: \mathbf{A} \succ 0
    \end{cases}
\end{equation}
Домножим на 2 и сложим первые два неравенства, третье раскроем.

\begin{equation}
    \begin{cases}
        2x^T A x + 2y^T A y + 2b^T (x + y) + 4 c \leq 0 \\
        x^T A x - x^T A y - y^T A x + y^T A y \geq 0 \\
    \end{cases}
\end{equation}
Вычтем из первого второе неравенство.

$$ x^T A x + y^T A y + x^T A y + y^T A x + 2b^T (x + y) + 4c \leq 0$$
$$ (x + y)^T A (x + y) + 2b^T (x + y) + 4c \leq 0$$
$$ \left( \frac{x + y}{2} \right)^T A \left( \frac{x + y}{2} \right) + b^T \left( \frac{x + y}{2} \right) + c \leq 0$$
Отсюда следует что $\frac{x + y}{2} \in \mathcal{C}$ следовательно $\mathcal{C}$ выпуклое.

\section*{Task 3}
$M := \left\{\mathbf{X} \mathbf{X}^{\top} \mid \mathbf{X} \in \mathbb{R}^{n \times k}, \operatorname{rank}(\mathbf{X})=k\right\}$.\\
$A \in M$ тогда $a \in \mathbb{R}^{n \times n}, \: \mathrm{rank} (A) = k$. \\
$X^T X = U \Sigma V^T V \Sigma U^T = U \Sigma^2 U^T = U \Sigma^2 U^{-1}$ - ЖНФ,   
следовательно $M$ - множество матриц $\mathbb{R}^{n \times n}$ с $k$ ненулевыми положительными собственными значениями.\\
Докажем, что коническая оболочка таких матриц - матрицы с $k$ и более ненулевыми положительными значениями. \\

\subsection*{$\Rightarrow$}
Докажем по индукции, что если мы можем получить $r$ произвольных положительных собственных значений
то можем и $r + 1$.\\
Пусть искомая матрица имеет вид $U \mathrm{diag} (\lambda_1, \ldots \lambda_{r + 1}, 0, \ldots 0) U^{-1}$.\\
Возьмем 
$A = U \mathrm{diag} \left(\lambda_1 , \frac{\lambda_2}{2}, \ldots, \frac{\lambda_{r}}{2}, 0, \ldots 0 \right) U^{-1}$, 
$B = U \mathrm{diag} \left(0, \frac{\lambda_2}{2}, \ldots, \frac{\lambda_{r}}{2}, \lambda_{r + 1}, 0, \ldots 0 \right) U^{-1}$.\\
Тогда $A + B = U \mathrm{diag} (\lambda_1, \ldots \lambda_{r + 1}, 0, \ldots 0) U^{-1}$.
Следовательно любая матрица с положительными собственными значениями ранга $k$ и выше лежит в конической оболочке $M$.

\subsection*{$\Leftarrow$}
Рассмотрим произвольную сумму $A + B = U_1 \Sigma_1^2 U_1^{-1} + U_2 \Sigma_2^2 U_2^{-1}$. \\
\begin{itemize}
    \item Так как $A$ и $B$ неотрицательно определенные очевидно что и сумма их неотрицательно определена, 
    значит сумма этих матриц имеет неотрицательные собственные значения.
    \item Т.к. $A$ и $B$ положительно определены то $\mathrm{rank}(A + B) \geq k$.
\end{itemize}

 
\end{document}