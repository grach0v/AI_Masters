\documentclass[12pt]{exam}
\usepackage{amsthm}
\usepackage{libertine}
\usepackage[utf8]{inputenc}
\usepackage[margin=1in]{geometry}
\usepackage{amsmath,amssymb}
\usepackage{multicol}
\usepackage[shortlabels]{enumitem}
\usepackage{siunitx}
\usepackage{cancel}
\usepackage{graphicx}
\usepackage{pgfplots}
\usepackage{listings}
\usepackage{tikz}
\usepackage{setspace}
\usepackage{mathtools}

\usepackage[T2A]{fontenc}
\usepackage[utf8]{inputenc}
\usepackage[russian]{babel}


\pgfplotsset{width=10cm,compat=1.9}
\usepgfplotslibrary{external}
\tikzexternalize

\newcommand{\class}{AI Masters Optimization} 
\newcommand{\examnum}{Homework 3} 
\newcommand{\examdate}{\today} 
\newcommand{\timelimit}{}
\newcommand{\pluseq}{\mathrel{+}=}
\newcommand{\minuseq}{\mathrel{-}=}
\doublespacing


\begin{document}
\pagestyle{plain}
\thispagestyle{empty}

\noindent
\begin{tabular*}{\textwidth}{l @{\extracolsep{\fill}} r @{\extracolsep{6pt}} l}
\textbf{\class} & \textbf{Name:} & \textit{Грачев Денис}\\
\textbf{\examnum} &&\\
\textbf{\examdate} &&\\
\end{tabular*}\\
\rule[2ex]{\textwidth}{2pt}
% ---

\section*{Сопряженные функции}
\subsection*{Task 1}
$$
    f(x) = \sum_{i=1}^n x_i \log \left( \frac{x_i}{1^T x} \right) 
$$
$$
    f^*(y) = \sup_x \left( (y, x) - f(x) \right) = 
    \sup_x \sum_{i=1}^n x_i \left( y_i - \log \left( \frac{x_i}{1^T x} \right) \right)
$$
Рассмотрим $x = (t, t, \ldots t)$, тогда 
$$
    f^*(y) \geq \sup_t \sum_{i=1}^n t \left( y_i - \log \left( \frac{t}{nt} \right) \right) = 
    t \sum_{i=1}^n (y_i + \log(n)) = \begin{cases}
        + \infty \:|\:  \sum_{i=1}^n (y_i + \log(n)) > 0, t \rightarrow + \infty \\
        + \infty \:|\:  \sum_{i=1}^n (y_i + \log(n)) < 0, t \rightarrow - \infty \\
        0 \:|\: \sum_{i=1}^n (y_i + \log(n)) = 0
    \end{cases}
$$
Пусть $\sum_{i=1}^n (y_i + \log(n))$ = 0, тогда рассмотрим $x = (2t, t, \ldots, t)$.\\
Аналогично, в зависимости от суммы, $f^*(y) \geq \begin{cases} +\infty \\ 0 \end{cases}$,
но так как обе суммы не могут одновременно равняться 0, то $f^*(y) \geq + \infty \Rightarrow f^*(y) = +\infty$.

% \begin{align*}
%     f(x) 
%         &= \sum_{i=1}^n x_i \log \left( \frac{x_i}{1^T x} \right) \\
%         &= \sum_{i=1}^n x_i \log(x_i) - x_i 1^T x \\
%         &= \left( \sum_{i=1}^n x_i \log(x_i) \right) - 1^T x \left( \sum_{i=1}^n x_i \right) \\
%         &= - (1^T x)^2 + \sum_{i=1}^n x_i \log(x_i) \\
%         &= \sum_{i=1}^n x_i \log(x_i) - \left( \sum_{i=1}^n x_i \right)^2
% \end{align*}
% Тогда 
% $$\nabla f(x)_i = \log(x_i) + 1 - 2 \sum_{i=j}^n x_j$$
% $$f^*(y) = \sup_x \left( (y, x) - f(x) \right)$$
% Так как функция непрерывная и диффернцируемая, то 
% $$
% x_0: f^*(y) = (y, x_0) - f(x_0) \Rightarrow 
% \nabla (y, x_0) - f(x_0) = 0 \Rightarrow 
% y = \nabla f(x_0)
% $$
% Подставим градиент
% $$ 
% y_i = \nabla f(x_0)_i = \log(x_{0i}) + 1 - 2\sum_{j=1}^n x_j
% $$
% Вычтем из всех уравнений первое
% $$ 
% y_i - y_1 =  \log(x_{0i}) - \log(x_{01})
% $$
% Выразим $x_{0i}$ через $y$ и $x_{01}$
% $$
% x_{0i} = e^{ y_i - y_1} x_{01}
% $$
% Рассмотрим первое уравнение
% $$
% y_1 = \log(x_{01}) + 1 - 2 \sum_{j=1}^n  e^{ y_j - y_1} x_{01}
% $$

\subsection*{Task 2}
Рассмотрим интуитивную геометрическую интерпритацию. \\
$f(x) = \max_{k=1, \ldots, p} (a_i x + b_i)$ - максимум из нескольких прямых то есть выпуклая функция. \\
$yx$ - прямая проходящая через 0 с коэфициентом $y$. \\
Следовательно функция $f_y(x) = xy - f(x)$ - вогнутая. \\
Пусть $a_m = \max_i a_i < y > 0$, тогда легко видеть что 
$(xy - f(x)) \xrightarrow[x \rightarrow +\infty]{} +\infty$, 
следовательно $f^*(y) = + \infty$. \\
Пусть $a_m = \max_i a_i = y > 0$, тогда $f^*(y) = -b_m$. \\
Рассмотрим крайнюю точку $x_m$ преломления функции $f(x)$. \\
Тогда $\forall x \geq x_m: xy - f(x) = -b_m$. \\
$\forall x < x_m: f(x) > a_m x_m + b_m \:\:,\:\:  yx - f(x) < yx - a_m x_m - b_m = -b_m$.\\
Следовательно $f^*(y) = -b_m$.\\
Пусть $a_m = \max_i a_i > y > 0$. 
Аналогично предыдущему случаю, но с поправкой что $\forall x \geq x_m: xy - f(x) \leq -b_m$ \\
Следовательно $f^*(y) = -b_m$. Аналогичен случай с $y < 0$, но $a_m = \min_i a_i$\\
Итого $f^*(y) = \begin{cases}
    + \infty \: | \: y > \max_i a_i, y > 0 \\
    -b_m \: | \: y \leq a_m = \max_i a_i, y > 0 \\ 
    + \infty \: | \: y > \min_i a_i, y < 0 \\
    -b_m \: | \: y \leq a_m = \min_i a_i, y < 0 \\
\end{cases}$

\subsection*{Task 3}
$g(x) = \inf_{x_1 + \ldots x_2 = x} (f_1(x_1) + \ldots f_k(x_k))$, 
где $f_i$ - выпуклые функции.
\begin{align*}
    g^*(y) 
        &= \sup_x (y, x) - g(x) \\
        &= \sup_x (y, x) - \inf_{x_1 + \ldots + x_2 = x} (f_1(x_1) + \ldots f_k(x_k)) \\
        &= \sup_x -\inf_{x_1 + \ldots + x_2 = x} (-y, x) f_1(x_1) + \ldots f_k(x_k) \\
        &= \sup_x -\inf_{x_1 + \ldots + x_2 = x} f_1(x_1) - (y, x_1) + \ldots f_k(x_k) - (y, x_k) \\
        &= -\inf_x \inf_{x_1 + \ldots + x_2 = x} \hat{f_1}(x_1) + \ldots \hat{f_k}(x_k) \quad \textit{где} \quad \hat{f_i}(x) = f_i(x) - yx \\
        &= -\inf_{x_1, \ldots, x_2 } \hat{f_1}(x_1) + \ldots \hat{f_k}(x_k) \quad \textit{где} \quad \hat{f_i}(x) = f_i(x) - yx \\
        &= - \left( \inf_x \hat{f_1}(x) + \ldots \inf_x \hat{f_k}(x) \right)
\end{align*}

\subsection*{Task 4}
$$
    f(x) = \begin{cases}
        |x| - \frac{1}{2} \quad | \quad |x| > 1 \\
        \frac{1}{2} x^2 \quad | \quad |x| \leq 1 
    \end{cases}
$$
Прибегнем к геометрической интерпритации снова.\\
Так как картинка симметриная, будем рассматривать $y > 0$\\
Тогда для $|y| > 1: f*(y) = + \infty$.\\
Для $|y| \leq 1$ супремум достигается на отрезке $[-1, 1]$, поэтому 
рассмотрим функцию $r_y(x) = xy - \frac{1}{2} x^2$. 
Ее максимум достигается в точке $y$, следовательно $f*(y) = y$. \\
$$
f*(y) = \begin{cases}
    +\infty \quad | \quad |y| > 1 \\
    y \quad | \quad |y| \leq 1
\end{cases}
$$

\section*{Субдифференциал}

\subsection*{Task 1}
$$ y^T x = f(x) + f^*(y) \quad \mathrm{iff} \quad y \in \partial f(x) $$
По определению субдифференциала
\begin{align*}
    y \in \partial f(x) 
        &\Rightarrow \forall x' \in \mathrm{dom} f: f(x') \geq f(x) + (y, x' - x) \\
        &\Rightarrow \forall x' \in \mathrm{dom} f: f(x') - (y, x') \geq f(x) - (y, x) \\
        &\Rightarrow \forall x' \in \mathrm{dom} f: (y, x') - f(x') \leq (y, x) - f(x) \\
        &\Rightarrow (y, x) - f(x) = \sup_t ((y, t) - f(t)) = f*(y) \\
        &\Rightarrow (y, x) = f(x) + f^*(y)
\end{align*} 

\subsection*{Task 3}
$$ a \in \partial f(x) \Leftrightarrow \forall y \in \mathrm{dom} f: f(y) \geq f(x) + (a, y - x)$$
\begin{align*}
    f(y) 
        &= \| y \|_1 = \sum_{i=1}^n | y_i | \\
        &\geq f(x) + (a, y - x) \\
        &= \sum_{i=1}^n | x_i | + a_i y_i - a_i x_i
\end{align*}

Отсюда легко видеть, что $a_i = \begin{cases}
    -1 \quad | \quad x_i < 0 \\
    [-1, 1] \quad | \quad x_i = 0 \\
    1 \quad | \quad x_i > 0 \\
\end{cases}$

\subsection*{Task 2}
$$ a \in \partial f(x) \Leftrightarrow \forall y \in \mathrm{dom} f: f(y) \geq f(x) + (a, y - x)$$
\begin{align*}
    f(y) 
        &= \sup_{0 \leq t \leq 1} (y_1 + y_2 t + \ldots + y_n t^{n - 1}) \\
        &\geq f(x) + (a, y - x) \\
        &= \sum_{i=1}^n a_i(y_i - x_i) + \sup_{0 \leq t \leq 1} (x_1 + x_2 t + \ldots + x_n t^{n - 1}) \\
        &= \sup_{0 \leq t \leq 1} (a_1y_1 - a_1x_1 + x_1 + a_2y_2 - a_2x_2 + x_2 t + \ldots + a_ny_n - a_nx_n + x_n t^{n - 1})
\end{align*}


\end{document}