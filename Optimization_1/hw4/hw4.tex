\documentclass[12pt]{exam}
\usepackage{amsthm}
\usepackage{libertine}
\usepackage[utf8]{inputenc}
\usepackage[margin=1in]{geometry}
\usepackage{amsmath,amssymb}
\usepackage{multicol}
\usepackage[shortlabels]{enumitem}
\usepackage{siunitx}
\usepackage{cancel}
\usepackage{graphicx}
\usepackage{pgfplots}
\usepackage{listings}
\usepackage{tikz}
\usepackage{setspace}

\usepackage[T2A]{fontenc}
\usepackage[utf8]{inputenc}
\usepackage[russian]{babel}

\newenvironment{sqcases}{%
  \matrix@check\sqcases\env@sqcases
}{%
  \endarray\right.%
}
\def\env@sqcases{%
  \let\@ifnextchar\new@ifnextchar
  \left\lbrack
  \def\arraystretch{1.2}%
  \array{@{}l@{\quad}l@{}}%
}
\makeatother


\pgfplotsset{width=10cm,compat=1.9}
\usepgfplotslibrary{external}
\tikzexternalize

\newcommand{\class}{AI Masters Optimization} 
\newcommand{\examnum}{Homework 4} 
\newcommand{\examdate}{\today} 
\newcommand{\timelimit}{}
\newcommand{\pluseq}{\mathrel{+}=}
\newcommand{\minuseq}{\mathrel{-}=}
% \doublespacing


\begin{document}
\pagestyle{plain}
\thispagestyle{empty}

\noindent
\begin{tabular*}{\textwidth}{l @{\extracolsep{\fill}} r @{\extracolsep{6pt}} l}
\textbf{\class} & \textbf{Name:} & \textit{Денис Грачев}\\
\textbf{\examnum} &&\\
\textbf{\examdate} &&\\
\end{tabular*}\\
\rule[2ex]{\textwidth}{2pt}
% ---

\section*{Условия оптимальности}
\subsection*{Task 1}
$$ \min_x c^T x $$
$$\mathrm{s.t.}\: x^TAx \leq 1 \:\: \textit{где}\: A \in S^n_{++}$$
Задача выпуклая, условие Слейтра выполняется (легко найти $x*$ для которого ограничение будет строгое). \\
Запишем условие ККТ. 
$$ L(x, \mu) = c^T x + \mu (x^T A x - 1)$$
$$ L'_x(x, \mu) = c + \mu (A + A^T)x = c + 2 \mu A x = 0$$
$$ \mu A x = \frac{-c}{2}$$
$$ x = A^{-1}\frac{-c}{2\mu}$$
Запишем условие дополняющей нежесткости
$$ \mu (x^T A x - 1) = 0$$
\subsubsection*{Пусть $\mu = 0$}
Тогда для $c \neq 0: L'_x (x, \mu) \neq 0$, следовательно минимум не достигается.
\subsubsection*{Пусть $x^T A x = 1$}
Тогда домжножим слева на $x^T$ равенство 
$$ \mu x^T A x = x^T \frac{-c}{2}$$
$$ \mu = x^T \frac{-c}{2}$$
$$ c x^T = -2\mu$$
$$ c^T x = -2\mu$$
Подставим $x$ из Лангражиана
$$ c^T  A^{-1}\frac{-c}{2\mu} = -2\mu$$
$$ 4\mu^2 = c^T A^{-1} c$$
$$ \mu = \frac{1}{2} \sqrt{c^T A^{-1}} c$$
Следовательно $x = \frac{-c A^{-1}}{\sqrt{c^T A^{-1} c}}$

\subsection*{Task 2}
$$ \min_{X \in S_{++}^n} \mathrm{trace}(X) - \log \det X$$
$$ \mathrm{s.t. } Xz = y \:\:\textit{ где }\:\: y \in \mathbb{R}^n, x\in \mathbb{R}^n, y^Tz= 1 $$
% Заметим, что последнее условие равносильно $z^TXz = 1$.\\
% $\Rightarrow$
% Пусть $Xz=y$ и $y^Tz=1$. Домжножим слева на $z^T$, тогда $z^TXz = z^Ty = 1$\\
% $\Leftarrow$ 
% Пусть $z^TXz = 1$ и $y^Tz=1$
Задача выпуклая, условие Слейтера выполняется, потому что найдется $X^*$ для которого равенство выполянется.\\
Запишем условие ККТ.\\
$$L(X, \lambda, \mu) = \mathrm{trace}(X) - \log \det X + \sum_{i=1}^n \lambda_i (X_i z - y_i) = \mathrm{trace}(X) - \log \det X + \lambda (Xz - y)$$
% при этом $\lambda_i (X_i z - y_i) = 0$
$$L(X, \lambda, \mu)'_X = I - {X^{-1}}^T + \begin{bmatrix} X_{ij}z_j \lambda_i \end{bmatrix} = 0$$
Будем искать среди диагональных матриц. Тогда 
\begin{equation}
    \begin{cases}
    1 - X_{ii}^{-1} + X_{ii}z_i\lambda_i = 0\\
    X_{ii}z_i = y_i 
    \end{cases}
\end{equation}
\begin{equation}
    \begin{cases}
    1 - \frac{z_i}{y_i} + y_i\lambda_i = 0\\
    X_{ii} = \frac{y_i}{z_i} 
    \end{cases}
\end{equation}
Возьмем $\lambda_i = \frac{z_i}{y_i^2}$. 
Тогда равенство выполняется. Следовательно минимум
$$ \sum_{i=1}^n \frac{y_i}{z_i} - \log \prod \frac{y_i}{z_i} = 0$$
Пусть $X \in S_{++}^n$ не диагональная. 
Представим ее в виде $X = S^{-1}DS$, где $D$ диаогнальная 
Тогда 
\begin{align*}
    f(X) 
        &= \mathrm{trace}(X) - \log \det X \\ 
        &= \mathrm{trace}(S^{-1}DS) - \log \det (S^{-1}DS) \\ 
        &= \mathrm{trace}(D) - \log \det D \\
        &= f(D)        
\end{align*}
Следовательно можно рассматривать только диагональные матрицы.

\subsection*{Task 3}
Найти вектор минимальной евклидовой нормы из выпуклой оболчки $a_1, \ldots, a_k$.\\
Будем считать, что вектора $a_1, \ldots a_k$ линейно независимы 
(всегда можно выбрать линейно независмый набор, порождающий такое же выпуклое множество).\\
Обоозначим $\begin{bmatrix} a_1 \\ \ldots \\ a_k \end{bmatrix} = A^{-1}$.
% $$ \| \min_x \begin{bmatrix} a_1 \\ \ldots \\ a_k \end{bmatrix} x \|_1$$
% $$ \textit{s.t. } 0 \leq x_i \leq 1 $$
% $$ \sum_{i=1}^n x_i = 1$$
$$ \min_y \| y \|_2$$
$$ \textit{s.t. } y = A^{-1} w \Rightarrow w = A y$$
$$ 0 \leq w_i \leq 1 \Leftrightarrow 0 \leq A_i y \leq 1 $$
$$ \sum_{i=1}^n w_i = 1 \Leftrightarrow \sum_{i=1}^n A_i y = 1 $$
Задача выпуклая, условие Слейтера выполнено. \\
Минимизировать норму, тоже самое что минимизровать квадрат нормы.\\
Запишем условие ККТ.
$$ 
L(y, \mu, \lambda) = 
\| y \|_2^2 + 
\sum_{i=1}^n (- \mu_i A_i y) + 
\sum_{i=1}^{n} (\mu_{i+n} A_i y - 1) + 
\lambda \left(1 - \sum_{i=1}^n A_i y \right)
$$
Условие $\leq 1$ можно убрать, оно автоматически будет 
выполняться из неотрицательности и суммы 1. 
$$ 
L(y, \mu, \lambda) = 
\| y \|_2^2 + 
\sum_{i=1}^n (- \mu_i A_i y) + 
\lambda \left(1 - \sum_{i=1}^n A_i y \right)
$$

\begin{align*}
    0 = L(y, \mu, \lambda)'_{y_j} 
        &= 
            2y_j + 
            \sum_{i=1}^n (- \mu_i A_{ij} ) + 
            \lambda \left(-\sum_{i=1}^n A_{ij} \right) \\
        &= 
            2y_j -
            \sum_{i=1}^n (\mu_i A_{ij} + \lambda A_{ij}) \\
    \Rightarrow y_j 
        &= \frac{\sum_{i=1}^n (\mu_i A_{ij} + \lambda A_{ij})}{2} \\
        &= \frac{\mu A_{:,j} + \overrightarrow{\lambda} A_{:,j}}{2} \\
    \Rightarrow y 
        &= \frac{\mu A + \overrightarrow{\lambda} A}{2} \\
        &= \frac{(\mu + \overrightarrow{\lambda}) A }{2} 
\end{align*}
Далее необходимо учесть
условие дополняющей нежесткости $\mu_i A_i y = 0$ и
допустимость значений. 
Но что-то у меня не особо дошло до ответа. 

\subsection*{Task 4}
$$ \min_x -\sum_{i=1}^n \log (\alpha_i + x_i) $$
$$ \textit{s.t. } \sum_{i=1}^n x_i = 1$$
$$ x_i \geq 0$$
Задача выпуклая, условие Слейтера выполняется. \\
Запишем условие ККТ. \\
$$ 
L(x, \mu, \lambda) = 
-\sum_{i=1}^n \log(\alpha_i + x_i) + 
\sum_{i=1}^n -\mu_i x_i + 
\lambda \left(1 - \sum_{i=1}^n x_i \right)
$$
$$ 
L(x, \mu, \lambda)'_{x_j} = 
-\frac{1}{\alpha_j + x_j} -\mu_j -\lambda = 0  
$$
$$
x_j = -\frac{1}{\mu_j + \lambda} - \alpha_j
$$
Так как 
\begin{equation*}
    \begin{cases}
        \sum_{i=1}^n x_i = \sum_{i=1}^n -\frac{1}{\mu_i + \lambda} - \alpha_i = 1 \\
        \mu_i x_i = -\frac{\mu_i}{\mu_i + \lambda} - \alpha_i = 0        
    \end{cases}
\end{equation*}
То из уравнений дополняющей нежесткости можно выразить $\mu_i$ через $\lambda, \alpha_i$,
но так как уравнение квадратное, то возможно два варианта. 
После чего будем подставлять все возможные варианты в 1 уравнение которое откуда можно найти $\lambda$ и соответсвенно все остальное.
Так как количество вариантов $2^n$ а каждая проверка занимет $n$ операций, то итоговая сложность $O(n2^n)$. 

\subsection*{Task 5}
$$ \min_{x} \| x - y \|_2^2 $$
$$ \textit{s.t. } \sum_{i=1}^n x_i = 1$$
$$ x_i \geq 0$$
Задача выпуклая, условие Слейтера выполняется. \\
Запишем условие ККТ.\\
$$ 
L(x, \mu, \lambda) = 
\| x - y \|_2^2 + 
\sum_{i=1}^n -\mu_i x_i + 
\lambda \left(1 - \sum_{i=1}^n x_i\right)
$$
\begin{align*}
    0 = L(x, \mu, \lambda)'_{x_j} 
        &= 2(x_j - y_j) -\mu_j x_j - \lambda \\
        &= x_j (2 - \mu_j) - 2y_j - \lambda \\
    \Rightarrow
        x_j &= \frac{2 y_j + \lambda}{2 - \mu_j}
\end{align*} 
Запишем ограничение и условие дополняющей нежесткости
\begin{equation*}
    \begin{cases}
        \sum_{i=1}^n x_i = \frac{2 y_j + \lambda}{2 - \mu_j} = 1 \\
        \mu_i x_i = \mu_i\frac{2 y_i + \lambda}{2 - \mu_i} = 0
    \end{cases}
\end{equation*}
\begin{equation*}
    \begin{cases}
        \sum_{i=1}^n x_i = \frac{2 y_i + \lambda}{2 - \mu_i} = 1 \\
        \mu_i (2 y_i + \lambda) = 0
    \end{cases}
\end{equation*}
Выберем один из $2^n$ вариантов где либо $\mu_i = 0$ либо $(2y_i + \lambda)$.
Тогда мы учтем условие дополняющей нежесткости и при подстановке в первое уравнение у нас останется 1 неизвестная $\lambda$ которую мы сможем найти.
Таким образом снова сложность $O(n 2^n)$. 
Что довольно странно, 
так как казалось бы геометрически нам необходимо 
просто спроецировать вектор на соответсвующую плоскость 
и если он не попал в ограничения, взять соответсвующую точку на границе.

\section*{Двойственные задачи}
\subsection*{Task 1}
$$ \min_{x} c^T x$$
$$ \textit{s.t } x_i (1 - x_i) = 0$$
$$ Ax \leq b$$
Запишем Лангражиан
$$
g(\mu, \lambda) = \inf_{x \in D} L(x, \mu, \lambda) = 
c^T x + 
\sum_{i=1}^n \mu_i (A_ix - b_i) + 
\sum_{i=1}^n \lambda_i x_i (1 - x_i)
$$
$$
L(x, \mu, \lambda)'_{x_j} = 
c_j + \sum_{i=1}^n \mu_i A_{ij} + \lambda_j (1 - 2x_j) = 0
$$
$$ 
x_j = \frac{c_j + \sum_{i=1}^n \mu_i A_{ij} + \lambda_j}{2}
$$
Следовательно двойственная задача 
$$
\max_{\mu, \lambda} L([\frac{c_j + \sum_{i=1}^n \mu_i A_{ij} + \lambda_j}{2} \textit{ for j in range(n)}], \mu, \lambda)
$$
$$
\textit{s.t. } \mu_i \geq 0
$$


\end{document}